\part{Introduction}


\chapter{WordPress}

\section{Introduction}

WordPress is web software you can use to create a beautiful website or blog. We like to say that WordPress is both free and priceless at the same time.

WordPress是一个注重美学、易用性和网络标准的个人信息发布平台。WordPress虽为免费的开源软件,但其价值无法用金钱来衡量。





WordPress started in 2003 with a single bit of code to enhance the typography of everyday writing and with fewer users than you can count on your fingers and toes. Since then it has grown to be the largest self-hosted blogging tool in the world, used on millions of sites and seen by tens of millions of people every day.

The core software is built by hundreds of community volunteers, and when you're ready for more there are thousands of plugins and themes available to transform your site into almost anything you can imagine. Over 60 million people have chosen WordPress to power the place on the web they call "home"-- we'd love you to join the family.

WordPress 是一款专注于易用性、运行速度和用户体验的发布软件。 WordPress 受益于一个活跃的社区,那是开源软件的灵魂所在。



\fbox{
	\parbox[c][82pt][t]{360pt}{
	“WordPress 的目的是为了创建一个优雅的,良好构架的基于 PHP 和 MySQL 以及 GPL 许可协议下的个人发布系统。它是b2/cafelog博客平台的正式继承者。WordPress 是新鲜的软件,但是它的根源和开发可以追溯到2001年。它是一款成熟而且稳定的产品。我们希望能把精力放在Web标准和用户体验上从而创造出一个与众不同的工具。”
	}
}

Besides the technical terminology of WordPress, it is also interesting to know the history of the name, WordPress. The name "WordPress" was originally coined by Christine Selleck in response to developer Matthew Mullenweg's desire to associate his new software project with printing presses. In this sense, press refers to the world of reporters, journalists, columnists, and photographers. An aptly chosen name, because WordPress serves as the printing press that enables its users to publish their words.

Everything you see here, from the documentation to the code itself, was created by and for the community. WordPress is an Open Source project, which means there are hundreds of people all over the world working on it. (More than most commercial platforms.) It also means you are free to use it for anything from your cat's home page to a Fortune 500 web site without paying anyone a license fee and a number of other important freedoms.

WordPress started as just a blogging system, but has evolved to be used as full content management system and so much more through the thousands of \href{http://wordpress.org/plugins/}{plugins and widgets} and \href{http://wordpress.org/themes/}{themes}, WordPress is limited only by your imagination. (And tech chops.)

使用WordPress可以搭建功能强大的网络信息发布平台,但更多的是应用于个性化的博客。针对博客的应用,WordPress能让您省却对后台技术的担心,集中精力做好网站的内容。

Ready to get started\footnote{WordPress is also available in 中文(简体).}? 

The latest and greatest WordPress, version 3.6, is now live to the world and includes a beautiful new blog-centric theme, bullet-proof autosave and post locking, a revamped revision browser, native support for audio and video embeds, and improved integrations with Spotify, Rdio, and SoundCloud. 

In addition to online resources like \href{http://wordpress.org/support/}{the forums} and \href{http://codex.wordpress.org/Mailing_Lists}{mailing lists} a great way to get involved with WordPress is to \href{http://wordcamp.org/}{attend or volunteer at a WordCamp}, which are free or low-cost events that happen all around the world to gather and educate WordPress users, organized by WordPress users. \href{http://wordcamp.org/}{Check out the website}, there might be a WordCamp near you.

{It's} Easy As
\begin{compactenum}
\item Find a Web Host and get great hosting while supporting WordPress at the same time.
\item Download \& Install WordPress with our famous 5-minute installation. Feel like a rock star.
\item Read the Documentation and become a WordPress expert yourself, impress your friends.
\end{compactenum}

The following is a list of some of the features that come standard with WordPress, however there are literally tens of thousands of plugins that extend what WordPress does, so the actual functionality is nearly limitless. You are also free to do whatever you like with the WordPress code, extend it or modify in any way or use it for commercial projects without any licensing fees. That is the beauty of free software, free meaning not only price but also the freedom to have complete control over it.

\subsection{Proven}

WordPress powers nearly a quarter of new sites today, is the content management system (CMS) of choice for more than two thirds of the top million sites making it the most popular on the web, and is trusted by content publishers both large and small including CNN and the NY Times. With more than 50 million sites globally and eight years of proven history, you know you’re getting the best software for the job.

\subsection{Easy to use}

At the core of WordPress is a dumb-simple interface similar to the desktop publishing software you use today. With no coding experience or expert knowledge necessary, the learning curve is often about as short as typing in your site’s URL and logging in. In fact, most users are able to pick up the basics without any training at all. Interfaces are polished and easy to use, and are the result of years of refinement. It’s the power of Microsoft Word with the intuitiveness of an iPhone.

\subsection{Built for Publishing}

WordPress makes sharing content and attracting readers to your site a breeze. Whether pushing content to social networks, ensuring that your website is provided in the optimal format to appear at the top of search results the moment you hit publish, or providing visitors the ability to subscribe to specific content sub-feeds in their favorite feed reader (or even via e-mail), WordPress is not simply a website, but rather a content-publishing platform. With a single click, you have a powerful megaphone to broadcast your message to the world.

\subsection{Backed by Community Support}

WordPress is supported by a vibrant community of users who have already solved many of the toughest challenges to sharing information today. The latest version of WordPress has been downloaded more than 10 million times since it was released a few months ago, and the prior version was downloaded more than 6,072,599 times. With a library of more than 20,000 free, open-source plug-ins and themes growing each day, and hundreds of core contributors each release cycle, the WordPress community is an ecosystem built around the platform’s viability and proven success.

\section{Content is King}

\subsection{Your Entire Workflow}

WordPress can take the place of your entire workflow from the initial draft to the time you hit publish – spelling, grammar, collaboration, and review – there’s no need for e-mails back-and-forth or expensive desktop software.

\subsection{Beyond Black and White}

Everything that makes webpages feel rich – pictures, videos, music, documents – can feel right at home in WordPress. With a drag-and-drop file uploader that uses the latest technology to ensure your file effortlessly makes it to the web page every time, and a media browser to help you store, organize and find the files you’re looking for, WordPress hosts the files that make your pages pop.

\subsection{Distraction Free Writing}

Between E-Mail, IMs, Texts, Tweets, and Status Updates, we have enough distractions as is today. Your publishing platform should not be one of them. While writing WordPress literally fades away letting you concentrate on your ideas themselves, not how you’re getting them out there.

\subsection{Never Lose a Word}

WordPress automatically saves your work as you type so you don’t have to worry if your computer crashes or you make a mistake. Want to go back to a previous version? Not a problem. Every time you hit save, WordPress creates a snapshot that you can restore with a single click.

\subsection{Time Travel}

Okay, not really, but it’s pretty close. WordPress lets you schedule posts for some time in the future or lets you backdate a post for some time in the past so that you can write when its convenient for you.

\subsection{Publish Anywhere}

The internet’s everywhere, so why shouldn’t your workflow be? WordPress has mobile applications for Android, iOS, Blackberry, Nokia, Windows Phone 7, even WebOS. Wherever you are, control of your site is literally at your fingertips. Your phone not listed? No fret. You can even post to your site by e-mail.

\subsection{Password protection}

You can give passwords to individual posts to hide them from the public. You can also have private posts which are viewable only by their author.

\subsection{Multi-paged posts}

If your post is too long, cut it up into pages, so your readers don't have to scroll to the end of the world.

\subsection{Save Drafts}

Save your unfinished articles, improve them later, publish when you're done.

\subsection{Previewing Posts}

Before you press the "Publish" button, you can look at the preview for the article you just wrote to check if everything is the way you want it. In fact, you can do that at any time, since the preview is "live".

\section{Iteratively Update}

\subsection{The Right Tuxedo for Any Content}

Your content deserves the best. The site should conform to your content, not the other way around. WordPress comes with a full theme system which makes designing everything from the simplest site to the most complicated portal a piece of cake. Have a new design every day. Your ideas should look as good as what they say.

\subsection{Obsessively Organized}

It doesn’t matter how much content you have, if your visitors can’t find it. WordPress organizes your content by day, by month, by year, by author, by category — any way you can describe it — and dynamically creates browsable archives so things always stay up to date.

\subsection{Killer Search Inside and Out}

WordPress has killer search baked in. Every word you write is fully searchable through a single box at the top of each page and if your users choose to use an external search engine like Google, rest assured, WordPress will present your content in a way that all but ensures it makes it to the top of the results every time.

\subsection{Even the URLs are Beautiful}

Ever go to a site and look up to the URL bar only to see a string of letter and number gobbledygook? WordPress realizes that websites are built for people, not computers. Every URL is intuitively written for humans and describes what your content says, not where it sits in a database.

\subsection{Typography Nerds Rejoice}

To do it right, publishing on the web can be a pain without the right tools. Every time you hit publish WordPress silently typesets each and every letter for seamless web production. Where many other CMSs let the details fall by the wayside, WordPress uses the Texturize engine to intelligently convert web-unfriendly characters like quotes, apostrophes, ellipses, em and en dashes, multiplication symbols, and ampersands into typographically correct HTML entities. For information about the proper use of such entities see Peter Sheerin's article The Trouble With Em ’n En.

\subsection{Bienvenidos a WordPress}

WordPress has been translated to more than 60 different languages, so however you say “publish”, you’ll be saying it in no time flat. You can create a site that is localized to your choice, and delivered in a language of your choice. The gettext method is used to translate and localize WordPress to the fullest extent.

\subsection{Drag and Drop Administration}

Weren’t a computer science major in college? No degree required, promise. Most of what users see from menus to the dynamic functionality on each page can be fully customized with simple drag-and-drop controls on the back end.

\subsection{Multiple Personalities}

Got a bunch of users? Not a problem. WordPress lets you define different roles for different users – just like in real life – and lets you assign privileges accordingly. Users can register themselves (if you want), and can submit content for your review.

\section{User Tested, Geek Approved}

\subsection{Out-of-the-box Power}

WordPress provides extensive functionality right out-of-the-box and often little customization is needed to adapt the software for your unique use. Many other CMSs rely on you to hunt down, install, and configure a long-list of add-ons just to get many of the features WordPress considers core (comments, RSS feeds, revisions, etc.) and relies on developers to undertake significant coding efforts to provide the functionality you need. WordPress does the heavy lifting so you don’t have to. Why reinvent the wheel when you already have the best wheel in the world?

\subsection{Open and Transparent}

WordPress is built by a dedicated community of professional developers, academics, and enthusiasts with the source code released to the world to take apart, build upon, and improve. It’s hallmark is a rapid development cycle, meaning frequent updates and always up-to-date software, all with no licensing fees or direct costs. And with an extensive international community professional support is always wherever you are.

\subsection{It's Your Data}

Some publishing platforms like to lock you in with proprietary data formats. Not here. WordPress relies on open standards to allow you to take your data with you, and even comes with tools to seamlessly import from many popular sources. It’s your data, and you should do what you want with it. We currently have importers for Movable Type, Textpattern, Greymatter, Blogger, and b2. Work on importers for Nucleus and pMachine is under way.

\subsection{It's Your Software}

WordPress is designed to be installed on your own web server, in the cloud, or in a shared hosting account. You have complete control. Unlike commercial software or third-party hosted services, you can be sure of being able to access and modify everything related to your site. You can even install WordPress on your personal computer, or on a corporate intranet.

\subsection{Power One Site or Millions}

WordPress offers multi-site technology. It is the same technology that powers over 20 million sites on WordPress.com and global sites like CNN and the New York Times. Multi-site technology allows users to have full administrative control over their own site, without any security concerns. Each site can have its own look-and-feel (themes), its own functionality (plug-ins), and manage its own users, while at the same time, network-wide policies and security updates can be deployed at the click of a button.

\subsection{Dynamic page generation}

No rebuilding of all your pages each time you update your site, or any aspect of it. All pages are generated using the database and the templates each time a page from your site is requested by a viewer. This means that updating your site, or its design is as fast as possible, and required server storage space usage is minimal.

\subsection{Template Driven Design}

WordPress uses templates to generate the pages dynamically. You can control the presentation of content by editing the templates using your favorite text-editor or IDE, or even the built-in Template Editor tool. Template tags make it easier to design the content and information displayed on your site. You don't need to be a PHP whiz to make your site's look-and-feel match your vision.


\section{A Serious Platform for Serious Content}

\subsection{Your Site is Your Castle}
 
WordPress has more than eight years of history powering stable, secure websites. Vulnerabilities are discovered quickly because of the wide user-base and dedicated open-source community, patches are rapidly developed by the dedicated security team, and often released in the span of hours from the time they are reported. WordPress comes with an integrated core-update system, so patches are deployed at the click of a mouse. WordPress sanitizes all user input, restricts URL access, has an extensive user permissioning system, and never stores passwords in an unencryptable format. WordPress uses WordPress.com’s 20 million users to beta test releases before they come out, so that by the time new versions are released, stakeholders can be confident in their stability.

\subsection{Extensive APIs}

WordPress’s core relies on its own extensive API interface (commonly known as dogfooding) which consequently allows developers to quickly and effectively customize the application to their unique needs. Many aspects of the essential WordPress experience can be overridden or modified by user-generated hooks and filters. These APIs help WordPress integrate seamlessly with existing systems, a necessity in a stove-pipe rich environments.

\subsection{Enterprise Ready}

Any challenge the organization faces, chances are, someone else has already tackled it and provided the code free of charge. WordPress has been adapted to countless enterprise environments, and provides support for Active Directory authentication, user management, work-flow integration, and scheduled backups, among other enterprise-centric features.

\subsection{Interoperability}

Want to connect WordPress to another system? WordPress uses XML-RPC, an open XML standard that allows different systems in different environments to talk to one another. XML-RPC is designed to be as simple as possible, while at the same time allowing for complex tasks to be performed. WordPress also supports an extended version of the Blogger API, MetaWeblog API, and finally the MovableType API. You can even use clients designed for other platforms like Zempt.

\subsection{Maintenance so simple you’ll wonder why you did it the other way}

Installing and upgrading WordPress is a piece of cake. WordPress’s famous five minute install is the envy of the industry, and with one click updates, you’ll know you are always using the best. Try it and you’ll wonder why all software isn’t this easy.

\subsection{Trust, but Verify}

Not everyone is evil, but keep those who are in check by limiting which html tags are kosher on your weblog. The default html tags allowed by WordPress are a sane choice to let people use html in their comments and posts, without compromising the safety of your data or server.

\subsection{Standards Compliant}

The WordPress team has gone to great lengths to ensure every bit of WordPress generated code is in full compliance with the standards of the W3C. This is important not only for interoperability with today’s browser but also for forward compatibility with the tools of the next generation. Your web site is a beautiful thing, and you should demand nothing less.

\subsection{More than a Blog}
 
Anyone who says WordPress is a mere blogging platform is covering for the fact that they haven’t been following the CMS’s explosive growth over the past couple years. Saying WordPress is only a blogging platform is like saying BMW is only a propeller manufacturer. In fact, the majority of the time, WordPress isn’t even used as a blog. With built in support for custom post types and custom taxonomies, if you can dream it, WordPress can make it a reality.


\section{Broadcast Your Ideas}


\subsection{Feeds}

The RSS 1.0 (aka RDF), RSS 2.0 and ATOM specifications are fully supported by WordPress, and what's more, just about any page on your site has an associated feed that your readers can subscribe to - there's a feed for the latest posts, for categories, comments, well, like we said earlier, for anything you want. The more options your readers have to keep track of different sections of your site, the easier it is for you to spread the word around the world. WordPress also fully supports RSS 2.0 with enclosures, so adding mp3 files (such as podcasts) to your RSS feeds is a snap.

\subsection{Inter-site Communication}

In an increasingly connected world, WordPress comes ready for PingBack and TrackBack, two very useful ways of connecting to other sites, and to enable them to do the same. Plus, WordPress supports pinging Ping-O-Matic, which means maximum exposure for your site to search engines.

\section{Grow Your Community}

\subsection{Community Building}
 
WordPress is not the YMCA, but it does help build communities around sites, through the use of comments, trackbacks and pingbacks, helping you keep in touch with the audience and fostering friendship

\subsection{Comments}

Visitors to your site can leave comments on individual entries, and through Trackback or Pingback can comment on their own site. You can enable or disable comments on a per-post basis.

\subsection{Spam protection}

Out of the box WordPress comes with very robust tools such as an integrated blacklist and open proxy checker to manage and eliminate comment spam on your blog, and there is also a rich array of plugins that can take this functionality a step further.

\subsection{Full user registration}

WordPress has a built-in user registration system that (if you choose) can allow people to register and maintain profiles and leave authenticated comments on your blog. You can optionally close comments for non-registered users. There are also plugins that hide posts from lower level users.

\subsection{Allowed html tag}

Not everyone is evil, but keep those who are in check by limiting which html tags are kosher on your site. The default html tags allowed by WordPress are a sane choice to let people use html in their comments, without compromising the safety of your data or server.

\subsection{Moderation}

For the control freak in all of us, WordPress provides an array of moderation options. You can moderate
\begin{compactitem}
\item all comments before they appear on the blog
\item comments with specific words in them
\item comments posted from specific IP addresses
\item comments containing more than some specified number of links.
\end{compactitem}

All these moderation options keep spammers and vandals in check.

\subsection{Notification}

WordPress can keep you in the loop by sending you an email each time there is a new comment or a comment awaiting moderation.

\section{License, Platform, and Philosoph}


\subsection{License}

WordPress is licensed under the GPLv2 or later which guarantees users several freedoms:
\begin{compactitem}
\item The freedom to run the program, for any purpose.
\item The freedom to study how the program works, and change it to make it do what you wish.
\item The freedom to redistribute.
\item The freedom to distribute copies of your modified versions to others.
\end{compactitem}

\subsection{Platform}

PHP (5.2.4 or newer) and MySQL (5.0.2 or newer) are required. 

\subsection{Philosophy}

WordPress's development is guided by a set of unwavering principles. They are:
\begin{compactitem}
\item Work Out of the Box
\item Design for the Majority
\item Decision not Options
\item Clean, Lean, and Mean
\item Strive for Simplicity
\item Deadlines are not Arbitrary
\item Be Mindful of the Vocal Minority
\item The WordPress Bill of Rights
\end{compactitem}


\section{WordPress Terms}

除了要了解博客软件WordPress是怎样运行的,还需要知道一些术语和概念。

\begin{compactitem}
\item 归档

博客也是在网上保存文章的好途径。多数博客以时间为基础将文章存档(例如以月或年归档),你还可设置在首页日历上显示每日存档。日志存档还可以以类别为基础,将同一类别日志存档在一起。

除此之外,也可用作者或字母表顺序给文章存档。存档方法多种多样。能这么悠闲轻松的组织和显示日志,正是博客成为流行的个人出版工具的原因之一。

\item Feeds

Feed是一种特殊的软件工具,它允许"Feed阅读器" 自动访问网站,查看最新内容,然后向其它网站发送更新资料。这使用户能够及时了解不同网站上发表的最新和最热门的信息。Feeds包括RSS(称为"丰富的站点摘要" 或者 "聚合内容"),Atom或RDF文件。Dave Shea,网络设计博客\href{http://mezzoblue.com/}{Mezzoblue}的作者对feed作过\href{http://www.mezzoblue.com/archives/2004/05/19/what_is_rssx/}{全面总结}。

\item 博客链接

博客链接是一个链接列表(可作分类),用来链接博客作者认为有价值或有趣的网页。链接的博客通常具有相同爱好。博客链接通常位于网页的"侧边栏",或独占一个页面。\href{http://blogrolling.com/}{BlogRolling}和\href{http://blo.gs/}{blo.gs}是提供与博客链接有关的功能和帮助的两个网站,它们帮助用户轻松维护链接并使其与博客融为一体。此外,WordPress也有内置的\href{http://codex.wordpress.org/Links_Manager}{链接管理},这样用户就不需要依赖第三方来创建和管理博客链接了。

\item 聚合内容

feed 是机器可读的 (通常是XML)博客定期更新内容。许多网络博客有feed (通常是 RSS, 但也可能是Atom 和RDF等等, 就像上面所描述的)。"feed阅读器"会不停地检测指定博客以了解其是否更新。如果博客已更新,它就会显示新日志的摘录(或者全部内容),并给其加上链接。Feed中也包括以前的日志,但当feed阅读器检查feed时,它真正要找的是新日志且它会自动发现新日志并下载下来供你阅读。这样只要把feed 链接添加到feed阅读器里,当博客新增文章时,Feed阅读器就会通知你,而你无需访问所有感兴趣的博客来看有无更新内容了。

\item 管理评论

博客中最令人兴奋的特征之一就是评论。这个高度交互式的功能允许用户评论日志,链接日志并向他人推荐日志——这就是trackbacks和pingbacks。我们也将讨论怎样审核和管理评论,及怎样处理博客中恼人的垃圾评论。

\item 固定链接

固定链接是个人博客的日志,分类和其它记录内容的永久URLs.其它博客作者浏览你的日志(或其它博客版块)时,你在邮件中输入日志链接时都会用到Permalink。由于他人也许会链接你的日志,所以日志的URL应保持不变,而固定链接正是永久性的(长时间有效)。

“漂亮的”固定链接源于这样一个理念:由于人们点击链接时,它的 URL 就会显示,因此我们希望显示的 URL 是有意义的,而不是充满了难以理解的参量。最好的固定链接应是“可修改的”,即用户可以更改浏览器中的文章链接,使其导航到博客的其它版块或列表。例如,以下是 WordPress 默认的固定链接: /index.php?p=423

用户怎会知道“p”代表什么?数字 423 来自哪儿呢?

相反,如果配置 WordPress,更改固定链接的话,拥有合理结构的漂亮的固定链接,链接就会链接到同一篇文章: /archives/2003/05/23/my-cheese-sandwich/

只要看一眼 URL,就能轻易看出链接包含了发表日志的日期和标题。或许有人也能想到把 URL改为/archives/2003/05/,就会得到 2003 年 5 月的所有日志列表。相当漂亮。

\item 使用电子邮件撰写博客

一些博客工具允许通过邮件把日志直接发送到博客上,而无需使用博客工具。WordPress就提供了这个较酷的功能。使用电子邮件,现在就可以将你的日志内容发送到预设的邮箱中。

\item 日志缩略名

如果使用了Permalinks, 日志缩略名(通常是日志标题)就是你的日志链接。博客软件可能会简化或截短你的标题,以让其更适合作链接。比如"I'll Make A Wish"这个标题可能会被截短为"ill-make-a-wish"。

在WordPress中,可以将日志缩略名缩略为其它形式,如"make-a- wish",这比“生病时许愿”听起来舒服多了。

\item 摘要

摘要是博客日志的简要概括,它有许多显示方式。在WordPress中, 摘要可以是特意撰写的文章总结性语句,或是日志前几段自动生成的文字,也可以是作者指定的某一句话。

\item 插件

插件是程序脚本上很酷的“饰品”,可用增强博客的已有功能,或添加新功能。

添加WordPress插件非常容易。在管理面板上就可以找到插件页面。从WordPress插件目录上下载插件后,在插件管理子面板上将它激活就可以了。当然,并不是所有的插件安装都很简单,但WordPress插件作者和开发者会使这个过程尽量简单。

\end{compactitem}

\section{WordPress Semantics}

As with many software packages, WordPress has its own lingo or jargon. This article will introduce you to some of the terminology used in WordPress.

\subsection{Introduction}

WordPress was created by the developers as weblogging or blogging software. A blog, as defined in the Codex Glossary, is an online journal, diary, or serial, published by a person or group of people. Many blogs are personal in nature, reflecting the opinions and interests of the owner. But, blogs are now important tools in the world of news, business, politics, and entertainment.

Blogs are a form of a Content Management System (CMS) which Wikipedia calls "a system used to organize and facilitate collaborative content creation." Both blogs and Content Management Systems can perform the role of a website (site for short). A website can be thought of as a collection of articles and information about a specific subject, service, or product, which may not be a personal reflection of the owner. More recently, as the role of WordPress has expanded, WordPress developers have begun using the more general term site, in place of blog.

\subsection{Terminology Related to Content}

The term Word in WordPress refers to the words used to compose posts. Posts are the principal element (or content) of a blog. The posts are the writings, compositions, discussions, discourses, musings, and, yes, the rantings of the blog's owner and guest authors. Posts, in most cases, are the reason a blog exists; without posts, there is no blog!

To facilitate the post writing process, WordPress provides a full featured authoring tool with modules that can be moved, via drag-and-drop, to fit the needs of all authors. The Dashboard QuickPress module makes it easy to quickly write and publish a post. There's no excuse for not writing.

Integral to a blog are the pictures, images, sounds, and movies, otherwise know as media. Media enhances, and gives life to a blog's content. WordPress provides an easy to use method of inserting Media directly into posts, and a method to upload Media that can be later attached to posts, and a Media Manager to manage those various Media.

An important part of the posting process is the act of assigning those posts to categories. Each post in WordPress is filed under one or more categories. Categories can be hierarchical in nature, where one category acts as a parent to several child, or grandchild, categories. Thoughtful categorization allows posts of similar content to be grouped, thereby aiding viewers in the navigation, and use of a site. In addition to categories, terms or keywords called tags can be assigned to each post. Tags act as another navigation tool, but are not hierarchical in nature. Both categories and tags are part of a system called taxonomies. If categories and tags are not enough, users can also create custom taxonomies that allow more specific identification of posts or pages or custom post types.

In turn, post categories and tags are two of the elements of what's called post meta data. Post meta data refers to the information associated with each post and includes the author's name and the date posted as well as the post categories. Post meta data also refers to Custom Fields where you assign specific words, or keys, that can describe posts. But, you can't mention post meta data without discussing the term meta.

Generally, meta means "information about"; in WordPress, meta usually refers to administrative-type information. So, besides post meta data, Meta is the HTML tag used to describe and define a web page to the outside world, like meta tag keywords for search engines. Also, many WordPress-based sites offer a Meta section, usually found in the sidebar, with links to login or register at that site. And, don't forget Meta Rules below:

The rules defining the general protocol to follow in using this Codex, or Meta, as in the MediaWiki namespace that refers to administrative functions within Codex. That's a lot of Meta!

After a post is made public, a blog's readers will respond, via comments, to that post, and in turn, authors will reply. Comments enable the communication process, that give-and-take, between author and reader. Comments are the life-blood of most blogs.

Finally, WordPress also offers two other content management tools called Pages and custom post types. Pages often present static information, such as "About Me", or "Contact Us", Pages. Typically "timeless" in nature, Pages should not be confused with the time-oriented objects called posts. Interestingly, a Page is allowed to be commented upon, but a Page cannot be categorized. A custom post type refers to a type of structured data that is different from a post or a page. Custom post types allow users to easily create and manage such things as portfolios, projects, video libraries, podcasts, quotes, chats, and whatever a user or developer can imagine.

\subsection{Terminology Related to Design}

The flexibility of WordPress is apparent when discussing terminology related to the design of a WordPress blog. At the core of WordPress, developers created a programming structure named The Loop to handle the processing of posts. The Loop is the critical PHP program code used to display posts. Anyone wanting to enhance and customize WordPress will need to understand the mechanics of The Loop.

Along with The Loop, WordPress developers have created Template Tags which are a group of PHP functions that can be invoked by designers to perform an action or display specific information. It is the Template Tags that form the basis of the Template Files. Templates (files) contain the programming pieces, such as Template Tags, that control the structure and flow of a WordPress site. These files draw information from your WordPress MySQL database and generate the HTML code which is sent to the web browser. A Template Hierarchy, in essence the order of processing, dictates how Templates control almost all aspects of the output, including Headers, Sidebars, and Archives. Archives are a dynamically generated list of posts, and are typically grouped by date, category, tag, or author.

Templates and Template Tags are two of the pieces used in the composition of a WordPress Theme. A Theme is the overall design of a site and encompasses color, graphics, and text. A Theme is sometimes called the skin. With the recent advances in WordPress, Theme Development is a hot topic. WordPress-site owners have available a long list of Themes to choose from in deciding what to present to their sites' viewers. In fact, with the use of a Theme Switcher Revisited Plugin, WordPress designers can allow their visitors to select their own Theme.

As the capabilities of WordPress have improved, developers have added tools that allow users to easily manage a site's look and functionality:
\begin{compactitem}
\item Widgets provide an easy way to add little programs, such as the current weather, to a sidebar.
\item Menus make it easy to define the navigation buttons that are typically present near the top of a site's pages.
\item The Background tool allows the user to change the background image and color of a site.
\item The Header tool gives the user control of the images displayed at the top of a site's various pages.
\item Formats allow the user to control the display of a specific post (i.e. display this post as an Aside or as a quote or as a gallery). The WordPress Twenty Twelve theme is an excellent example of a theme that uses these tools.
\end{compactitem}

And speaking of the WordPress Twenty Twelve theme, developers and users are encouraged to explore that theme in detail. The WordPress Twenty Twelve theme, developed by the WordPress community, demonstrates the use of tools such as Menus and Widgets, provides examples of recommended theme coding techniques, and emphasizes the use of the Child Theme concept to shield a theme from getting overwritten during a WordPress update.

Plugins are custom functions created to extend the core functionality of WordPress. The WordPress developers have maximized flexibility and minimized code bloat by allowing outside developers the opportunity to create their own useful add-on features. As evidenced by the Plugin Directory, there's a Plugin to enhance virtually every aspect of WordPress. A Plugin management tool makes it extremely easy to find and install Plugins.

\subsection{Terminology for the Administrator}

Another set of terms to examine are those involving the Administration of a WordPress site. A comprehensive set of Administration Panels enables users to easily administer and monitor their blog. A WordPress administrator has a number of powers which include requiring a visitor to register in order to participate in the blog, who can create new posts, whether comments can be left, and if files can be uploaded to the blog. An Administrator also defines Links and the associated Link Categories which are an important part of a blog's connection to the outside world.

Some of the main administrative responsibilities of a WordPress blog involve adding, deleting, and managing Registered Users. Administering users means controlling Roles and Capabilities, or permissions. Roles control what functions a registered user can perform as those functions can range from just being able to login at a blog to performing the role administrator.

Another chief concern for the blog administrator is Comment Moderation. Comments, also called discussions, are responses to posts left for the post author by the visitor and represent an important part of "the give and take" of a blog. But Comments must be patrolled for Spam and other malicious intentions. The WordPress Administration Comments SubPanel simplifies that process with easy-to-use screens which add, change, and delete Comments.

And not to be forgotten is the obligation for an administrator to keep their WordPress current to insure that the latest features, bugs, and security fixes are in effect. To accomodate administrators, WordPress has a simple Upgrade Tool to download and install the lastest version of WordPress. There's no excuse to not upgrade!

\subsection{The Terminology of Help}


The final set of jargon relates to helping you with WordPress. First and foremost is the hanging Help tab that is displayed under each of the Administration SubPanels. That contextual help describes the function and use of the current SubPanel and provides links to other help topics. And, there are other help resources available to WordPress users; Getting More Help, Finding WordPress Help, Troubleshooting, and WordPress FAQ (frequently asked questions) are good starting points. Also Getting Started with WordPress will jump-start readers into the world of WordPress and the excellent WordPress Lessons provide in-depth tutorials on many of the aspects of using WordPress. Among the most important resources is the WordPress Support Forum where knowledgeable volunteers answer your questions and help solve any problems related to WordPress. And, of course, this Codex which is filled with hundreds of articles designed to make your WordPress experience a success!




\section{WordPress Themes}

Once you are familiar with how WordPress works, it's time to get creative and start customizing. The tutorial now splits into different subjects that require no order. From here on you can do whatever you want, adding and subtracting, perfecting and scrambling your site at will. The amount of effort you put into the site is now up to you. You can work with the two WordPress Themes that came with the installation, or seek out another Theme that better meets your needs. You can totally customize all the links and information, or get serious and completely re-design the entire site to do whatever you want. You have the basics, the rest is up to your imagination.

WordPress allows you to change the look of your site using Themes. Themes are presentation styles that completely change the look of your site. Designed by WordPress users, there are hundreds of themes available for you to choose from. In your Presentation panel, you will see two themes, classic and default. To try this quick-change process, simply select Classic and then click View Site to see how it looks. Wow, you have another look and nothing else on the site has changed. It's that easy.

Go back to the Presentation panel (Back button on your browser) and select Default to bring the design back to what you had. To see it again, click View Site, and there it is. Honestly, it is that simple.

There are hundreds of WordPress Themes to choose from. All do basically the same thing but graphically present the information in a myriad of ways. Choose a few that look interesting to you, and meet your audience's needs and your desires, and then test drive them following the test drive instructions above. Click through the whole site, the categories and archives as well as the individual posts to see how the Theme handles each one. The look may be nice on the front page, but if it handles things in a way you don't like on the single post, then you will have to dig into the code and make changes. Not ready for that, try another theme.

When you are ready to plunge into the code, you can customize the look and layout of the site through CSS and modifing the Themes (or create your own).If you run into problems, check out the Codex's Troubleshooting Themes article.

If you are familiar with CSS, HTML, and even PHP and MySQL, consider customizing the Theme to your own needs. This is not for the timid, and it is for the informed and experienced. If you want to expand your web page design and development skills, WordPress can help:
\begin{compactitem}
\item \href{http://codex.wordpress.org/Using_Themes}{Using Themes}
\item \href{http://codex.wordpress.org/Theme_Development}{Theme Development}
\item \href{http://codex.wordpress.org/Stepping_Into_Templates}{Stepping Into Templates}
\item \href{http://codex.wordpress.org/Templates}{Templates Files}
\item \href{http://codex.wordpress.org/Blog_Design_and_Layout}{Blog Design and Layout}
\item \href{http://codex.wordpress.org/CSS}{CSS Overview, Tips, Techniques, and Resources}
\item \href{http://codex.wordpress.org/FAQ#Layout}{FAQ - WordPress Layout}
\item \href{http://codex.wordpress.org/Stepping_Into_Template_Tags}{Stepping Into Template Tags}
\item \href{http://codex.wordpress.org/Template_Tags}{Template Tags}
\item \href{http://codex.wordpress.org/CSS_Troubleshooting}{CSS Troubleshooting}
\item \href{http://codex.wordpress.org/CSS_Fixing_Browser_Bugs}{CSS Fixing Browser Bugs}
\end{compactitem}

\section{WordPress Plugins}

WordPress Plugins are also known as add-ons or extensions. They are software scripts that add functions and events to your website. They cover the gamut from up-to-date weather reports to simple organization of your posts and categories. Plugins are designed by volunteer contributors and enthusiasts who like challenges and problem solving. They are usually fairly simple to install through the WordPress Admin Plugin panel, just follow the instructions provided by the plugin author. Remember, these are free and non-essential. If you have any problems with plugins, contact the plugin author's website or plugin source first, then search the Internet for help with that specific plugin, and if you haven't found a solution, then visit the WordPress forums for more help.
\begin{compactitem}
\item \href{http://www.wp-plugins.org/}{WordPress Plugin Repository}
\item \href{http://www.wp-plugins.net/}{WordPress Plugins}
\item \href{http://codex.wordpress.org/Managing_Plugins}{Managing Plugins}
\item \href{http://codex.wordpress.org/Plugins}{Plugins}
\end{compactitem}

The exciting thing about WordPress is that there are few limits. Thousands of people are using WordPress for blogging and for running their websites. All have a different look and different functions on their sites.

What you do from here is up to you, but here are a few places to take that first step beyond the basics:
\begin{compactitem}
\item \href{http://codex.wordpress.org/WordPress_Features}{WordPress Features}
\item \href{http://codex.wordpress.org/Working_with_WordPress}{Working with WordPress}
\item \href{http://codex.wordpress.org/Pages}{Using Pages}
\item \href{http://codex.wordpress.org/The_Loop}{Understanding the WordPress Loop}
\item \href{http://codex.wordpress.org/Troubleshooting}{Troubleshooting}
\item \href{http://codex.wordpress.org/Using_Permalinks}{Using Permalinks}
\end{compactitem}

You are pouring over your feeds, viewing a website with an article or story that catches your eye, and you want to share that information with your WordPress site viewers. A quick trick is to use the WordPress Press It\footnote{This feature was removed from WordPress 2.5, and was replaced in WordPress 2.6 with the Press This function.} feature.

WordPress makes it easy to quickly add links and information to your site through the use of a bookmarklet called Press It. A bookmarklet looks like a link in your Favorites, Bookmarks, or Links list but it is much more powerful. It adds the capability to quickly create WordPress post entries on the fly while working on the Internet.

If you find something of interest on the Internet, you can click the Press It link and a window with your site's Administration Write Post panel will popup with the page you are viewing listed as a link. You can then write about the page, assign a category, and add any other information and then press Save and it will be immediately posted on your WordPress site.

The Press It bookmarklet is found at the bottom of the Write Posts panel. Drag and drop it onto your Favorites, Bookmarks, or Links list or toolbar.

To activate, simply click on the "Press It" bookmarklet link. A window will open with the URI of the current site displayed, and the site's title as your post title.

The link is Javascript, and its form is:

\begin{lstlisting}[language=HTML]
<a href="javascript:Q='';
if(top.frames.length==0)Q=document.selection.createRange().text;
void (btw=window.open ('$SiteURL/wp-admin/bookmarklet.php?text
='+escape(Q)+'&popupurl='+escape (location.href)+'&popuptitle='+
escape(document.title),'bookmarklet',&'scrollbars=yes,width=600,
height=460,left=100,top=150,status=yes'));btw.focus();">
Press it - $SiteName</a>
\end{lstlisting}

\verb|$SiteName| is the name of the installation, \verb|$SiteURL| is the installation directory.

Give it a try. It's really easy, and makes blogging and adding information to your WordPress site fast and easy.






\chapter{Weblog}

\section{Introduction}

“博客”是“网络博客”(weblog)的缩写,是指以时间为顺序记录信息并保持更新的网站。它通常是以日记形式撰写的个人主页,还包含其它网站的链接。从个人生活到政治话题都可成为博客内容,也就是说,它可以只涉及单个狭窄主题也可以包括所有主题范围。

很多博客都专注于单独的主题,比如网页设计、家庭理财、体育以及手机等无线技术。还有很多是关于作者的日常生活以及自己的思考。

一般来说(虽然有例外),博客包括以下内容:
\begin{compactitem}
\item 主要的内容区,其中文章按时间顺序排列,最新文章在最上面。通常还有文章类别。
\item 以前日志的存档
\item 评论区域
\item 其它网站的链接列表,有时称作"友情链接"。
\item 一个或者多个``feeds", 如RSS, Atom 或RDF 文件。
\end{compactitem}

当然,不排除有些博客还包括其它功能。

\section{Blog\&CMS}

CMS 或者"内容管理系统" 是用来管理网站的软件。大多数博客软件都是一种CMS,它们提供创建和维护博客功能,使人们在因特网上发表日志像写文章那样简单,只要给文章一个标题,并将标题放在(一个或者多个)类别下面就可开始编写内容了。CMS程序提供的功能较为庞大复杂,但博客工具让你仅凭直觉就能轻松的完成操作,而它则负责处理文章显示效果及公共访问这类后台工作。换句话说,你负责撰写,而博客工具则负责帮你管理网站!

WordPress致力于改善博客外观,提升博客性能,同时使它产生的html代码符合网络标准。它是一个高级的博客工具,提供了一套完整丰富的功能。在管理面板上,可设置显示效果及各种功能,轻松地撰写日志,然后按下按钮,就能将它立即发表到因特网上!

创建新博客确实很困难,许多人因此望而却步。也确有博客没有评论也无人访问。你想从千百万博客中脱颖而出,想让你的博客成为那几十万个真正被访问的博客之一吗?

下面的简单贴士会帮助你掌握博客技巧:

\begin{compactenum}
\item 定期发表日志,但如果没有什么值得发表的,就不要发表。
\item 坚持只谈论几个特别话题。
\item 不要在首页上放满了 '订阅'和 '给我投票'的链接,除非有人特别喜欢你的博客,以致可以忽视这些(但它们通常只会起负面作用) 。
\item 如果可能的话,尽量使用简明的主题。
\item 享受写博客的乐趣吧,记得给其他博客发表评论(因为他们一般会回访!)!
\end{compactenum}

\section{Posts}

对任何网站来说,内容是存在的理由。零售类网站以产品的目录为特色。大学的网站包含了他们的校园、课程和老师等信息。新闻类网站展示了最新的新闻故事。对一个个人博客来说,上面可能有一堆的意见或评论。如果没有内容被更新,我们根本没有理由访问一个站点两次。

在一个博客上,内容包括了作者写的文章(有时候也被叫做"posts"或者"entries")。的确,有的博客有多位作者,他们分别写各自的文章。通常情况下,博客作者在一个基于网络构建的博客系统中撰写文章。一些博客系统还支持使用独立的“博客客户端”软件,它允许作者离线写文章,稍后再上传。

\section{Comments}



什么是交互式网站?如果网站访客能够留下关于网站或文章的评论,小贴士,或感想,这不是更有趣吗?在博客上,就能这样做了!博客最令人兴奋的一点就是能在上面发表评论。

大多数博客能允许访客发表评论,有的甚至能让其它博客作者在不访问博客的情况下就能留下评论,以及链接日志并向他人推荐日志,这就是"pingbacks" 或者"trackbacks"。无论何时,只要在自己的文章中引用了其它站点的文章, trackback就会通知其它bloggers。这就加强了网站之间的交流和沟通。

\begin{compactenum}
\item Trackbacks

\href{http://codex.wordpress.org/Glossary#Trackback}{Trackbacks}最早是由\href{http://www.movabletype.org/}{MovableType}博客软件包的创建者——SixApart开发的。

\fbox{%
	\parbox[c][50pt][t]{350pt}{
简单来说, TrackBack是在网站间互通消息的工具:是甲向乙说话的一种方法,"你可能对这个有兴趣。"甲若要向乙表达这个句子,就要向乙发送一个 TrackBack ping。}%
}

下面是一个更好的解释:
\begin{compactitem}
\item 甲在博客上发表一篇日志。
\item 乙在甲的博客上评论,但也想让自己的访客看看评论内容,并让其能在自己的博客上评论。
\item 乙在自己的博客上发表日志,同时向甲的博客发送一个trackback。
\item 甲的博客收到了显示原始文章评论的trackback,且评论包含了乙的文章链接。
\end{compactitem}

这个方法使更多的人加入了讨论(甲和乙的读者都可以通过链接看到对方的文章),trackback有一定的真实性,因为它们来自另一个博客。不幸地是,引入的trackback并没有真正的真实性,它们的确可以伪造。

多数trackbacks只将乙所说的一小部分(称作"摘要")发送给甲,这个"内容摘要"鼓励他们点击到乙的站点阅读其余内容(可能是评论)。通常情况下,乙向甲的博客发送的trackback贴满了评论。当然,甲可以在他自己的服务器上编辑评论,但这也意味着"真实性"问题并没有真正得到解决。(注: 甲只能在他自己的服务器上改变trackback的内容。他不能编辑发送trackback的乙站点上的文章。)

SixApart已发布\href{http://www.movabletype.org/docs/mttrackback.html}{官方trackback说明}。

\item Pingbacks

Pingbacks的产生是为了解决trackback遗留的问题。 然而,官方的pingback文档使pingbacks听起来和trackbacks一样糟糕:

\fbox{%
	\parbox[c][65pt][t]{350pt}{
例如,Yvonne在她的博客上写了一篇有趣的文章。Kathleen阅读了Yvonne的文章,做了评论并链接了Yvonne的原始文章。使用 pingback,Kathleen的博客会自动通知Yvonne,她的博客已被链接,Yvonne的博客就会将这个信息加入到她的网站。	
	}%
}

pingbacks和trackbacks有三个重要区别:

\begin{compactitem}
\item Pingbacks和trackbacks使用了两种完全不同的信息技术(分别是XML-RPC和HTTP POST)。
\item Pingbacks支持自动察觉,软件能自动发现文章中的链接并自动尝试pingback这些URLs,然而使用Trackbacks时,必须手动输入 trackbacks要发送到的trackback URL.
\item Pingbacks不发送任何内容。
\end{compactitem}

理解pingbacks的最好办法是把它看作远程评论:

\begin{compactitem}
\item 甲在他的博客上发表一篇日志。
\item 乙在她自己的博客上发表日志并链接甲的文章。若双方都有激活的pingback,系统将自动给甲发送pingback.
\item 甲的博客收到了pingback,然后自动来到乙的文章来确认pingback是否确实源自那里。
\end{compactitem}

显示在甲博客上的Pingback就是乙博客的日志链接。这样,文章的所有编辑都由作者来控制(trackback摘要是可由接受者编辑的)。自动确认过程引进了一定的真实性,这就使伪造pingback变得更难了。

有些人觉得trackbacks较好,因为甲的读者至少可以看见乙所说的一些话,然后决定是否要阅读更多的内容(然后点击到乙的博客)。有的人则认为 pingbacks较好,因为它们在文章之间创建了一个可证实的连接。

\item 核实 Pingbacks和Trackbacks

经常有人批评博客中的评论缺少权威性,因为所有人都可以用他们喜欢的名字任意发表评论,但却没有核实过程来确认发表人就是页面上显示的评论人。而Trackbacks 和Pingbacks使这一状况有所改善,因为它们都可用来核实评论。

\item 审核评论

审核评论允许网站所有者和作者监控和管理日志评论,并有助于处理评论中的垃圾广告。可用它来审核评论,删除不受欢迎的评论,批准通过较好评论或对评论做出其它安排。

\item 垃圾评论

垃圾评论是指博客上发布的无用评论(或trackbacks,或pingbacks),通常与文章内容无关。它们可以包含一个或多个其它网站或域名链接。为使他们的域名在Google中获得更高的网页排名,发送垃圾广告的人把评论作为工具,那样的话,他们未来就可将域名以更高的价格出售或者在现有网站的搜索结果中获得更高的排名。

发送垃圾广告的人是毫不留情的;因为这涉及到真正的钱,他们努力"工作",甚至创建自动工具(机器人)迅速散发他们的垃圾广告。许多博客,特别是新手,常觉得网站简直被垃圾评论包围了。

然而,我们对此有解决办法!因为WordPress有许多用来打击垃圾评论的工具。预先稍作努力就可以控制垃圾评论,我们当然没理由为此放弃博客。


\end{compactenum}







\chapter{WordPress.org}

\section{Introduction}

On this site you can download and install a software script called WordPress. To do this you need a web host who meets the \href{http://wordpress.org/about/requirements/}{minimum requirements} and a little time. WordPress is completely customizable and can be used for almost anything. There is also a service called \href{http://wordpress.com/?ref=wporg-about}{WordPress.com} which lets you get started with a new and free WordPress-based blog in seconds, but varies in several ways and is less flexible than the WordPress you download and install yourself.

WordPress was born out of a desire for an elegant, well-architectured personal publishing system built on \href{http://php.net/}{PHP} and \href{http://mysql.com/}{MySQL} and licensed under the \href{http://www.gnu.org/copyleft/gpl.html}{GPLv2} (or later). It is the official successor of b2/cafelog. WordPress is fresh software, but its roots and development go back to 2001. It is a mature and stable product. We hope by focusing on user experience and \href{http://webstandards.org/}{web standards} we can create a tool different from anything else out there.

The license under which the WordPress software is released is the GPLv2 (or later) from the \href{http://www.fsf.org/}{Free Software Foundation}. A copy of the license is included with every copy of WordPress, but you can also read \href{http://wordpress.org/about/gpl/}{the text of the license} here.

Part of this license outlines requirements for derivative works, such as plugins or themes. Derivatives of WordPress code inherit the GPL license. \href{http://drupal.org/}{Drupal}, which has the same GPL license as WordPress, has an excellent page on \href{http://drupal.org/licensing/faq/}{licensing as it applies to themes and modules}(their word for plugins).

There is some legal grey area regarding what is considered a derivative work, but we feel strongly that plugins and themes are derivative work and thus inherit the GPL license. If you disagree, you might want to consider a non-GPL platform such as \href{http://www.s9y.org/}{Serendipity} (BSD license) or \href{http://habariproject.org/en/}{Habari} (Apache license) instead.

We've got a variety of resources to help you get the most out of WordPress. Your first stop should be our \href{http://codex.wordpress.org/}{documentation}, where you'll find information on everything from installing WordPress for the first time to creating your own themes and plugins.

For a bit more about WordPress' history \href{http://en.wikipedia.org/wiki/WordPress}{check out the WordPress Wikipedia page} or \href{http://codex.wordpress.org/History}{this page on our own Codex}.

\chapter{Requirement}

To run WordPress your host just needs a couple of things:
\begin{compactitem}
\item \href{http://www.php.net/}{PHP} version 5.2.4 or greater
\item \href{http://www.mysql.com/}{MySQL} version 5.0 or greater
\item \href{http://httpd.apache.org/docs/current/mod/mod_rewrite.html}{Apache mod\_rewrite module} Apache mod\_rewrite模块\footnote{可选,用于支持“固定链接”和“站点网络”功能。}
\end{compactitem}

The requirements have changed as of WordPress 3.2. The minimum requirements for WordPress 3.1 are PHP 4.3 and MySQL 4.1.2.

That’s really it. We recommend \href{http://httpd.apache.org/}{Apache} or \href{http://nginx.net/}{Nginx} as the most robust and featureful server for running WordPress, but any server that supports PHP and MySQL will do. That said, we can’t test every possible environment and \href{http://wordpress.org/hosting/}{each of the hosts on our hosting page} supports the above and more with no problems.

\textbf{Not required, but recommended for better security}

Hosting is more secure when PHP applications, like WordPress, are run using your account’s username instead of the server’s default shared username. The most common way nowadays for hosting companies to do this is using \href{http://www.suphp.org/}{suPHP}. Just ask your potential host if they run suPHP or something similar.

\chapter{Features}

WordPress powers more than 17\% of the web - a figure that rises every day. Everything from simple websites, to blogs, to complex portals and enterprise websites, and even applications, are built with WordPress.

WordPress combines simplicity for users and publishers with under-the-hood complexity for developers. This makes it flexible while still being easy-to-use.

The following is a list of some of the features that come as standard with WordPress, however there are literally thousands of plugins that extend what WordPress does, so the actual functionality is nearly limitless. You are also free to do whatever you like with the WordPress code, extend it or modify in any way or use it for commercial projects without any licensing fees. That is the beauty of free software, free refers not only to price but also the freedom to have complete control over it.

Here are some of the features that we think that you'll love.








